% Use only LaTeX2e, calling the article.cls class and 12-point type.

\documentclass[12pt]{article}

% Users of the {thebibliography} environment or BibTeX should use the
% scicite.sty package, downloadable from *Science* at
% http://www.sciencemag.org/authors/preparing-manuscripts-using-latex 
% This package should properly format in-text
% reference calls and reference-list numbers.

\usepackage{scicite}

\usepackage{times}

% The preamble here sets up a lot of new/revised commands and
% environments.  It's annoying, but please do *not* try to strip these
% out into a separate .sty file (which could lead to the loss of some
% information when we convert the file to other formats).  Instead, keep
% them in the preamble of your main LaTeX source file.


% The following parameters seem to provide a reasonable page setup.

\topmargin 0.0cm
\oddsidemargin 0.2cm
\textwidth 16cm 
\textheight 21cm
\footskip 1.0cm


%The next command sets up an environment for the abstract to your paper.

\newenvironment{sciabstract}{%
\begin{quote} \bf}
{\end{quote}}



% Include your paper's title here

\title{A simple {\it Science\/} Template} 


% Place the author information here.  Please hand-code the contact
% information and notecalls; do *not* use \footnote commands.  Let the
% author contact information appear immediately below the author names
% as shown.  We would also prefer that you don't change the type-size
% settings shown here.

\author
{John Smith,$^{1\ast}$ Jane Doe,$^{1}$ Joe Scientist$^{2}$\\
\\
\normalsize{$^{1}$Department of Chemistry, University of Wherever,}\\
\normalsize{An Unknown Address, Wherever, ST 00000, USA}\\
\normalsize{$^{2}$Another Unknown Address, Palookaville, ST 99999, USA}\\
\\
\normalsize{$^\ast$To whom correspondence should be addressed; E-mail:  jsmith@wherever.edu.}
}

% Include the date command, but leave its argument blank.

\date{}



%%%%%%%%%%%%%%%%% END OF PREAMBLE %%%%%%%%%%%%%%%%



\begin{document} 

% Double-space the manuscript.

\baselineskip24pt

% Make the title.

\maketitle 



% Place your abstract within the special {sciabstract} environment.

\begin{sciabstract}
  This document presents a number of hints about how to set up your
  {\it Science\/} paper in \LaTeX\ .  We provide a template file,
  \texttt{scifile.tex}, that you can use to set up the \LaTeX\ source
  for your article.  An example of the style is the special
  \texttt{\{sciabstract\}} environment used to set up the abstract you
  see here.
\end{sciabstract}



% In setting up this template for *Science* papers, we've used both
% the \section* command and the \paragraph* command for topical
% divisions.  Which you use will of course depend on the type of paper
% you're writing.  Review Articles tend to have displayed headings, for
% which \section* is more appropriate; Research Articles, when they have
% formal topical divisions at all, tend to signal them with bold text
% that runs into the paragraph, for which \paragraph* is the right
% choice.  Either way, use the asterisk (*) modifier, as shown, to
% suppress numbering.

\section*{Introduction}

In this file, we present some tips and sample mark-up to assure your
\LaTeX\ file of the smoothest possible journey from review manuscript
to published {\it Science\/} paper.  We focus here particularly on
issues related to style files, citation, and math, tables, and
figures, as those tend to be the biggest sticking points.  Please use
the source file for this document, \texttt{scifile.tex}, as a template
for your manuscript, cutting and pasting your content into the file at
the appropriate places.

{\it Science\/}'s publication workflow relies on Microsoft Word.  To
translate \LaTeX\ files into Word, we use an intermediate MS-DOS
routine \cite{tth} that converts the \TeX\ source into HTML\@.  The
routine is generally robust, but it works best if the source document
is clean \LaTeX\ without a significant freight of local macros or
\texttt{.sty} files.  Use of the source file \texttt{scifile.tex} as a
template, and calling {\it only\/} the \texttt{.sty} and \texttt{.bst}
files specifically mentioned here, will generate a manuscript that
should be eminently reviewable, and yet will allow your paper to
proceed quickly into our production flow upon acceptance \cite{use2e}.

\section*{Formatting Citations}

Citations can be handled in one of three ways.  The most
straightforward (albeit labor-intensive) would be to hardwire your
citations into your \LaTeX\ source, as you would if you were using an
ordinary word processor.  Thus, your code might look something like
this:


\begin{quote}
\begin{verbatim}
However, this record of the solar nebula may have been
partly erased by the complex history of the meteorite
parent bodies, which includes collision-induced shock,
thermal metamorphism, and aqueous alteration
({\it 1, 2, 5--7\/}).
\end{verbatim}
\end{quote}


\noindent Compiled, the last two lines of the code above, of course, would give notecalls in {\it Science\/} style:

\begin{quote}
\ldots thermal metamorphism, and aqueous alteration ({\it 1, 2, 5--7\/}).
\end{quote}

Under the same logic, the author could set up his or her reference list as a simple enumeration,

\begin{quote}
\begin{verbatim}
{\bf References and Notes}

\begin{enumerate}
\item G. Gamow, {\it The Constitution of Atomic Nuclei
and Radioactivity\/} (Oxford Univ. Press, New York, 1931).
\item W. Heisenberg and W. Pauli, {\it Zeitschr.\ f.\ 
Physik\/} {\bf 56}, 1 (1929).
\end{enumerate}
\end{verbatim}
\end{quote}

\noindent yielding

\begin{quote}
{\bf References and Notes}

\begin{enumerate}
\item G. Gamow, {\it The Constitution of Atomic Nuclei and
Radioactivity\/} (Oxford Univ. Press, New York, 1931).
\item W. Heisenberg and W. Pauli, {\it Zeitschr.\ f.\ Physik} {\bf 56},
1 (1929).
\end{enumerate}
\end{quote}


That's not a solution that's likely to appeal to everyone, however ---
especially not to users of B{\small{IB}}\TeX\ \cite{inclme}.  If you
are a B{\small{IB}}\TeX\ user, we suggest that you use the
\texttt{Science.bst} bibliography style file and the
\texttt{scicite.sty} package, both of which are downloadable from our author help site.
{\bf While you can use B{\small{IB}}\TeX\ to generate the reference list, please don't submit 
your .bib and .bbl files; instead, paste the generated .bbl file into the .tex file, creating
 \texttt{\{thebibliography\}} environment.}
 You can also
generate your reference lists directly by using 
\texttt{\{thebibliography\}} at the end of your source document; here
again, you may find the \texttt{scicite.sty} file useful.

Whatever you use, be
very careful about how you set up your in-text reference calls and
notecalls.  In particular, observe the following requirements:

\begin{enumerate}
\item Please follow the style for references outlined at our author
  help site and embodied in recent issues of {\it Science}.  Each
  citation number should refer to a single reference; please do not
  concatenate several references under a single number.
\item The reference numbering  continues from the 
main text to the Supplementary Materials (e.g. this main 
text has references 1-3; the numbering of references in the 
Supplementary Materials should start with 4). 
\item Please cite your references and notes in text {\it only\/} using
  the standard \LaTeX\ \verb+\cite+ command, not another command
  driven by outside macros.
\item Please separate multiple citations within a single \verb+\cite+
  command using commas only; there should be {\it no space\/}
  between reference keynames.  That is, if you are citing two
  papers whose bibliography keys are \texttt{keyname1} and
  \texttt{keyname2}, the in-text cite should read
  \verb+\cite{keyname1,keyname2}+, {\it not\/}
  \verb+\cite{keyname1, keyname2}+.
\end{enumerate}

\noindent Failure to follow these guidelines could lead
to the omission of the references in an accepted paper when the source
file is translated to Word via HTML.



\section*{Handling Math, Tables, and Figures}

Following are a few things to keep in mind in coding equations,
tables, and figures for submission to {\it Science}.

\paragraph*{In-line math.}  The utility that we use for converting
from \LaTeX\ to HTML handles in-line math relatively well.  It is best
to avoid using built-up fractions in in-line equations, and going for
the more boring ``slash'' presentation whenever possible --- that is,
for \verb+$a/b$+ (which comes out as $a/b$) rather than
\verb+$\frac{a}{b}$+ (which compiles as $\frac{a}{b}$).  
 Please do not code arrays or matrices as
in-line math; display them instead.  And please keep your coding as
\TeX-y as possible --- avoid using specialized math macro packages
like \texttt{amstex.sty}.

\paragraph*{Tables.}  The HTML converter that we use seems to handle
reasonably well simple tables generated using the \LaTeX\
\texttt{\{tabular\}} environment.  For very complicated tables, you
may want to consider generating them in a word processing program and
including them as a separate file.

\paragraph*{Figures.}  Figure callouts within the text should not be
in the form of \LaTeX\ references, but should simply be typed in ---
that is, \verb+(Fig. 1)+ rather than \verb+\ref{fig1}+.  For the
figures themselves, treatment can differ depending on whether the
manuscript is an initial submission or a final revision for acceptance
and publication.  For an initial submission and review copy, you can
use the \LaTeX\ \verb+{figure}+ environment and the
\verb+\includegraphics+ command to include your PostScript figures at
the end of the compiled file.  For the final revision,
however, the \verb+{figure}+ environment should {\it not\/} be used;
instead, the figure captions themselves should be typed in as regular
text at the end of the source file (an example is included here), and
the figures should be uploaded separately according to the Art
Department's instructions.








\section*{What to Send In}

What you should send to {\it Science\/} will depend on the stage your manuscript is in:

\begin{itemize}
\item {\bf Important:} If you're sending in the initial submission of
  your manuscript (that is, the copy for evaluation and peer review),
  please send in {\it only\/} a PDF version of the
  compiled file (including figures).  Please do not send in the \TeX\ 
  source, \texttt{.sty}, \texttt{.bbl}, or other associated files with
  your initial submission.  (For more information, please see the
  instructions at our Web submission site.)
\item When the time comes for you to send in your revised final
  manuscript (i.e., after peer review), we require that you include
   source files and generated files in your upload. {\bf The .tex file should include
the reference list as an itemized list (see "Formatting citations"  for the various options). The bibliography should not be in a separate file.}  
  Thus, if the
  name of your main source document is \texttt{ltxfile.tex}, you
  need to include:
\begin{itemize}
\item \texttt{ltxfile.tex}.
\item \texttt{ltxfile.aux}, the auxilliary file generated by the
  compilation.
\item A PDF file generated from
  \texttt{ltxfile.tex}.

\end{itemize}
\end{itemize}

% Your references go at the end of the main text, and before the
% figures.  For this document we've used BibTeX, the .bib file
% scibib.bib, and the .bst file Science.bst.  The package scicite.sty
% was included to format the reference numbers according to *Science*
% style.

%BibTeX users: After compilation, comment out the following two lines and paste in
% the generated .bbl file. 

\bibliography{scibib}

\bibliographystyle{Science}





\section*{Acknowledgments}
Include acknowledgments of funding, any patents pending, where raw data for the paper are deposited, etc.

%Here you should list the contents of your Supplementary Materials -- below is an example. 
%You should include a list of Supplementary figures, Tables, and any references that appear only in the SM. 
%Note that the reference numbering continues from the main text to the SM.
% In the example below, Refs. 4-10 were cited only in the SM.     
\section*{Supplementary materials}
Materials and Methods\\
Supplementary Text\\
Figs. S1 to S3\\
Tables S1 to S4\\
References \textit{(4-10)}


% For your review copy (i.e., the file you initially send in for
% evaluation), you can use the {figure} environment and the
% \includegraphics command to stream your figures into the text, placing
% all figures at the end.  For the final, revised manuscript for
% acceptance and production, however, PostScript or other graphics
% should not be streamed into your compliled file.  Instead, set
% captions as simple paragraphs (with a \noindent tag), setting them
% off from the rest of the text with a \clearpage as shown  below, and
% submit figures as separate files according to the Art Department's
% instructions.


\clearpage

\noindent {\bf Fig. 1.} Please do not use figure environments to set
up your figures in the final (post-peer-review) draft, do not include graphics in your
source code, and do not cite figures in the text using \LaTeX\
\verb+\ref+ commands.  Instead, simply refer to the figure numbers in
the text per {\it Science\/} style, and include the list of captions at
the end of the document, coded as ordinary paragraphs as shown in the
\texttt{scifile.tex} template file.  Your actual figure files should
be submitted separately.

\end{document}




















